% file : FECoreStruct.text
% author : Alicia Wang, Total SA
% This template shows the structure of FECore of the open source software FEBio.


\documentclass[landscape,11pt]{report}
\usepackage{fancyhdr}
\usepackage[utf8]{inputenc}

 \usepackage{tikz}

% \usepackage{graphicx}
% \usepackage{tikzexternal}

\usepackage[left=27mm, right=27mm, top=30mm, bottom=25mm]{geometry}

\pagestyle{fancy}
\fancyhf{}
\fancyhead[C]{\textcolor[gray]{0.4}{Structure of the namespace FECore}}
\fancyfoot[L]{\textcolor[gray]{0.4}{\footnotesize Alicia \textsc{Wang} - March 2012} }
\fancyfoot[R]{\thepage}

\setlength{\headheight}{14pt}

\author{Alicia \sc{Wang}}
\date\today
\title{Structure of the namespace FECore}


\pgfrealjobname{FECoreStruct}
\usetikzlibrary{trees}

%%-------------------------
\begin{document}


\tikzstyle{selected} = [draw=red,fill=red!30]
\tikzstyle{optional} = [dashed,fill=gray!50]

\tikzstyle{abstract} = [draw=red, dashed, text centered, text width = 32mm,
fill=red!30, anchor=west, font=\sffamily]
\tikzstyle{concret} = [rectangle, draw=black, thick, text centered, text width = 40mm,
anchor=west, font=\sffamily]
\tikzstyle{general} = [magenta, very thick, font=\sffamily\large]


\begin{figure*}[!h]

\beginpgfgraphicnamed{pic1}
\begin{tikzpicture}[%
    matrixchild/.style={grow=down,anchor=west,
	grow via three points={one child at (0.5,-1) and
	two children at (0.5,-1) and (0.5,-2)},
	edge from parent path ={(\tikzparentnode.south) |- (\tikzchildnode.west)}},
	solverchild/.style={grow=down,anchor=east, xshift = -54mm,
	grow via three points={one child at (0.5,-1) and
	two children at (0.5,-1) and (0.5,-2)},
	edge from parent path={(\tikzparentnode.south) |- (\tikzchildnode.east)}}]

%Equation system part
	\node at (0.5,-0.5) (SparseMP) [concret, text width = 32mm] {SparseMatrixProfile};

	\node (SparseM) [abstract, below of = SparseMP, node distance = 28mm] {SparseMatrix}
		child [matrixchild] { node (CompactM) [abstract] {CompactMatrix}
			child [matrixchild] { node (CompactSymmM) [concret] {CompactSymmMatrix} }
			child [matrixchild] { node (CompactUnSymmM) [concret] {CompactUnSymmMatrix} }
			}
		child [matrixchild] { node (DenseM) [concret, below of = CompactUnSymmM] {DenseMatrix} }
		child [matrixchild] { node (SkylineM) [concret, below of = DenseM] {SkylineMatrix}};

	\path[->] (SparseMP) edge [orange]
		       node [above,sloped] {support} (SparseM);
	\draw[<-] (SparseM.south) |- (CompactM.west);
	\draw[<-] (CompactM.south) |- (CompactSymmM.west);

	\node (NonLinearS)
	[abstract, text width = 40mm, below of = SkylineM, node distance = 4cm]
	{NonLinearSystem};


% Solver part
	\node (LinearS) [abstract, right of = SparseMP, node distance = 16cm] {LinearSolver}
		child [solverchild]{ node (ConjGradIterS) [concret] {ConjGradIterSolver} }
		child [solverchild]{ node (PSLDLTS) [concret] {PSLDLTSolver} }
		child [solverchild]{ node (RCICGS) [concret] {RCICGSolver} }
		child [solverchild]{ node (WSMPS) [concret] {WSMPSolver} }
	    child [solverchild]{ node (PardisoS) [concret] {PardisoSolver} }
		child [solverchild]{ node (SuperLUS) [concret] {SuperLUSolver} }
		child [solverchild]{ node (SuperLUMTS) [concret] {SuperLU\_MT\_Solver} }
		child [solverchild]{ node (LUS) [concret] {LUSolver} }
		child [solverchild]{ node (SkylineS) [concret] {SkylineSolver} };

	\draw[<-] (LinearS.south) |- (ConjGradIterS.east);

	\node (BFGSS) [concret, below of = SkylineS, node distance = 28mm] {BFGSSolver};

	\draw[->,orange] (LinearS.east) .. controls +(up:6mm) and +(right:16mm) .. node[above,sloped]
					 {support} (BFGSS.east);

% Solving relation
% 	\draw[<-,blue] (CompactSymmM.east) -- node[above,sloped] {solve} (ConjGradIterS.west);
% 	\draw[<-,blue]	(CompactSymmM.east) -- (PSLDLTS.west);
% 	\draw[<-,blue]	(CompactSymmM.east) -- (RCICGS.west);
% 	\draw[<-,blue]	(CompactSymmM.east) -- (WSMPS.west);
% 	\draw[<-,blue]	(CompactSymmM.east) -- (PardisoS.west);
% 	\draw[<-,blue]	(CompactUnSymmM.east) -- (PardisoS.west);
% 	\draw[<-,blue]	(CompactUnSymmM.east) -- (SuperLUS.west);
% 	\draw[<-,blue]	(CompactUnSymmM.east) -- (SuperLUMTS.west);
% 	\draw[<-,blue]	(DenseM.east) -- (LUS.west);
% 	\draw[<-,blue]	(SkylineM.east) -- (SkylineS.west);
% 	\draw[<-,blue]	(NonLinearS.east) -- (BFGSS.west);


	\draw[magenta] (0,0) rectangle (10,-10.5);
	\node at (8,-0.5) [general] {Linear Systems};

	\draw[magenta] (0,-11.5) rectangle (10,-14);
	\node at (8,-13.5) [general] {Non Linear Systems};

	\draw[magenta] (12,0) rectangle (21,-10.5);
	\node at (14,-0.5) [general] {Linear Solvers};

	\draw[magenta] (12,-11.5) rectangle (21,-14);
	\node at (14,-13.5) [general] {Non Linear Solvers};

	\draw[gray!80,very thick] (11,0) -- (11, -15);
		\node at (8.5,-15) [general, gray, font=\sffamily\Large] {Equation Systems};
		\node at (12.5,-15) [general, gray, font=\sffamily\Large] {Solvers};

\end{tikzpicture}
\endpgfgraphicnamed

\end{figure*}

\newpage
\begin{figure*}[!h]

\beginpgfgraphicnamed{pic2}
\begin{tikzpicture}[%
    matrixchild/.style={grow=down,anchor=west,
	grow via three points={one child at (0.5,-1) and
	two children at (0.5,-1) and (0.5,-2)},
	edge from parent path ={(\tikzparentnode.south) |- (\tikzchildnode.west)}},
	solverchild/.style={grow=down,anchor=east, xshift = -54mm,
	grow via three points={one child at (0.5,-1) and
	two children at (0.5,-1) and (0.5,-2)},
	edge from parent path={(\tikzparentnode.south) |- (\tikzchildnode.east)}}]

%Equation system part
	\node at (0.5,-0.5) (SparseMP) [concret, text width = 32mm] {SparseMatrixProfile};

	\node (SparseM) [abstract, below of = SparseMP, node distance = 28mm] {SparseMatrix}
		child [matrixchild] { node (CompactM) [abstract] {CompactMatrix}
			child [matrixchild] { node (CompactSymmM) [concret] {CompactSymmMatrix} }
			child [matrixchild] { node (CompactUnSymmM) [concret] {CompactUnSymmMatrix} }
			}
		child [matrixchild] { node (DenseM) [concret, below of = CompactUnSymmM] {DenseMatrix} }
		child [matrixchild] { node (SkylineM) [concret, below of = DenseM] {SkylineMatrix}};

	\path[->] (SparseMP) edge [orange]
		       node [above,sloped] {support} (SparseM);
	\draw[<-] (SparseM.south) |- (CompactM.west);
	\draw[<-] (CompactM.south) |- (CompactSymmM.west);

	\node (NonLinearS)
	[abstract, text width = 40mm, below of = SkylineM, node distance = 4cm]
	{NonLinearSystem};


% Solver part
	\node (LinearS) [abstract, right of = SparseMP, node distance = 16cm] {LinearSolver}
		child [solverchild]{ node (ConjGradIterS) [concret] {ConjGradIterSolver} }
		child [solverchild]{ node (PSLDLTS) [concret] {PSLDLTSolver} }
		child [solverchild]{ node (RCICGS) [concret] {RCICGSolver} }
		child [solverchild]{ node (WSMPS) [concret] {WSMPSolver} }
	    child [solverchild]{ node (PardisoS) [concret] {PardisoSolver} }
		child [solverchild]{ node (SuperLUS) [concret] {SuperLUSolver} }
		child [solverchild]{ node (SuperLUMTS) [concret] {SuperLU\_MT\_Solver} }
		child [solverchild]{ node (LUS) [concret] {LUSolver} }
		child [solverchild]{ node (SkylineS) [concret] {SkylineSolver} };

	\draw[<-] (LinearS.south) |- (ConjGradIterS.east);

	\node (BFGSS) [concret, below of = SkylineS, node distance = 28mm] {BFGSSolver};

	\draw[->,orange] (LinearS.east) .. controls +(up:6mm) and +(right:16mm) .. node[above,sloped]
					 {support} (BFGSS.east);

% Solving relation
	\draw[<-,blue] (CompactSymmM.east) -- node[above,sloped] {solve} (ConjGradIterS.west);
	\draw[<-,blue]	(CompactSymmM.east) -- (PSLDLTS.west);
	\draw[<-,blue]	(CompactSymmM.east) -- (RCICGS.west);
	\draw[<-,blue]	(CompactSymmM.east) -- (WSMPS.west);
	\draw[<-,blue]	(CompactSymmM.east) -- (PardisoS.west);
	\draw[<-,blue]	(CompactUnSymmM.east) -- (PardisoS.west);
	\draw[<-,blue]	(CompactUnSymmM.east) -- (SuperLUS.west);
	\draw[<-,blue]	(CompactUnSymmM.east) -- (SuperLUMTS.west);
	\draw[<-,blue]	(DenseM.east) -- (LUS.west);
	\draw[<-,blue]	(SkylineM.east) -- (SkylineS.west);
	\draw[<-,blue]	(NonLinearS.east) -- (BFGSS.west);

\end{tikzpicture}
\endpgfgraphicnamed

\end{figure*}

\newpage

\begin{figure*}[!h]

\beginpgfgraphicnamed{pic3}

\begin{tikzpicture}[%
    matrixchild/.style={grow=down,anchor=west,
	grow via three points={one child at (0.5,-1) and
	two children at (0.5,-1) and (0.5,-2)},
	edge from parent path ={(\tikzparentnode.south) |- (\tikzchildnode.west)}},
	solverchild/.style={grow=down,anchor=east, xshift = -54mm,
	grow via three points={one child at (0.5,-1) and
	two children at (0.5,-1) and (0.5,-2)},
	edge from parent path={(\tikzparentnode.south) |- (\tikzchildnode.east)}}]

%Equation system part
	\node at (0.5,-0.5) (SparseMP) [concret, text width = 32mm] {SparseMatrixProfile};

	\node (SparseM) [abstract, below of = SparseMP, node distance = 28mm] {SparseMatrix}
		child [matrixchild] { node (CompactM) [abstract] {CompactMatrix}
			child [matrixchild] { node (CompactSymmM) [concret] {CompactSymmMatrix} }
			child [matrixchild] { node (CompactUnSymmM) [concret] {CompactUnSymmMatrix} }
			}
		child [matrixchild] { node (DenseM) [concret, below of = CompactUnSymmM] {DenseMatrix} }
		child [matrixchild] { node (SkylineM) [concret, below of = DenseM] {SkylineMatrix}};

	\path[->] (SparseMP) edge [orange]
		       node [above,sloped] {support} (SparseM);
	\draw[<-] (SparseM.south) |- (CompactM.west);
	\draw[<-] (CompactM.south) |- (CompactSymmM.west);

	\node (NonLinearS)
	[abstract, text width = 40mm, below of = SkylineM, node distance = 4cm]
	{NonLinearSystem};


% Solver part
	\node (LinearS) [abstract, right of = SparseMP, node distance = 16cm] {LinearSolver}
		child [solverchild]{ node (ConjGradIterS) [concret] {ConjGradIterSolver} }
		child [solverchild]{ node (PSLDLTS) [concret] {PSLDLTSolver} }
		child [solverchild]{ node (RCICGS) [concret] {RCICGSolver} }
		child [solverchild]{ node (WSMPS) [concret] {WSMPSolver} }
	    child [solverchild]{ node (PardisoS) [concret] {PardisoSolver} }
		child [solverchild]{ node (SuperLUS) [concret] {SuperLUSolver} }
		child [solverchild]{ node (SuperLUMTS) [concret] {SuperLU\_MT\_Solver} }
		child [solverchild]{ node (LUS) [concret] {LUSolver} }
		child [solverchild]{ node (SkylineS) [concret] {SkylineSolver} };

	\draw[<-] (LinearS.south) |- (ConjGradIterS.east);

	\node (BFGSS) [concret, below of = SkylineS, node distance = 28mm] {BFGSSolver};

	\draw[->,orange] (LinearS.east) .. controls +(up:6mm) and +(right:16mm) .. node[above,sloped]
					 {support} (BFGSS.east);

% Solving relation
	\draw[<-,blue] (CompactSymmM.east) -- node[above,sloped] {solve} (ConjGradIterS.west);
	\draw[<-,blue]	(CompactSymmM.east) -- (PSLDLTS.west);
	\draw[<-,blue]	(CompactSymmM.east) -- (RCICGS.west);
	\draw[<-,blue]	(CompactSymmM.east) -- (WSMPS.west);
	\draw[<-,blue]	(CompactSymmM.east) -- (PardisoS.west);
	\draw[<-,blue]	(CompactUnSymmM.east) -- (PardisoS.west);
	\draw[<-,blue]	(CompactUnSymmM.east) -- (SuperLUS.west);
	\draw[<-,blue]	(CompactUnSymmM.east) -- (SuperLUMTS.west);
	\draw[<-,blue]	(DenseM.east) -- (LUS.west);
	\draw[<-,blue]	(SkylineM.east) -- (SkylineS.west);
	\draw[<-,blue]	(NonLinearS.east) -- (BFGSS.west);


	\draw[magenta] (0,0) rectangle (10,-10.5);
	\node at (8,-0.5) [general] {Linear Systems};

	\draw[magenta] (0,-11.5) rectangle (10,-14);
	\node at (8,-13.5) [general] {Non Linear Systems};

	\draw[magenta] (12,0) rectangle (21,-10.5);
	\node at (14,-0.5) [general] {Linear Solvers};

	\draw[magenta] (12,-11.5) rectangle (21,-14);
	\node at (14,-13.5) [general] {Non Linear Solvers};

	\draw[gray!80,very thick] (11,0) -- (11, -15);
		\node at (8.5,-15) [general, gray, font=\sffamily\Large] {Equation Systems};
		\node at (12.5,-15) [general, gray, font=\sffamily\Large] {Solvers};

\end{tikzpicture}
\endpgfgraphicnamed

\end{figure*}

\end{document}